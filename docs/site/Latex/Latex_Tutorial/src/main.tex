\documentclass{article}
\usepackage[utf8]{inputenc}
\usepackage{xcolor}
\usepackage{ulem}
\usepackage{amsmath}


%set paragraph indentation
\setlength{\parindent}{1cm} 
\begin{document}
\pagecolor{darkgray}
% \pagecolor{lightgray}
% \color{blue}
\color{cyan}

% >>> INSERT TEXT HERE <<<x
\section{Text}
\subsection{Changing Text Sizes}
\begin{itemize}
    \item Normal text \Huge huge text \normalsize normal text
    \item {\Large Large text in braces} normal outside
\end{itemize}

\subsection{Font Styles}
\begin{itemize}
    \item textbf - \textbf{bold text}
    \item textit - \textit{italic text}
    \item  underline - \underline{underlined text}
    \item bold + italics + underline - \textbf{\textit{\underline{itlalized undelined bold text}}}
\end{itemize}




\LaTeX{} has problem with long underlines - it messes up spacing:
This longer underlined text \underline{will cause sonme troubles along the way with spacing and breaking lines}
so it is better to use package ulem wit option uline ssome textsome textsome textsome textsome textsome textsome textome text

\medskip

\LaTeX{} has problem with long underlines - it messes up spacing:
This longer underlined text \uline{will cause sonme troubles along the way with spacing and breaking lines}
so it is better to use package \texttt{ulem} with option uline ssome textsome 
textsome textsome textsome textsome textsome textome text

Also \texttt{ulem} has:
\begin{itemize}
    \item \uuline{double underline}
    \item \uwave{wavy underline}
\end{itemize}

\subsection{Text Emphasis}
\begin{itemize}
    \item Regular \normalem{emphasis}
    \item \textit{Italic \normalem{emphasis}}
    \item \normalem{Emphasized \normalem{emphasis} text}
\end{itemize}
ulem package messes up emphasis, to get it right we need to use \texttt{\char`\\normalem} to get it right
\subsection{Font Families}
\begin{itemize}
    \item \textrm{Default Roman text}
    \item \textsf{Sans serif text}
    \item \texttt{Typewriter text}
\end{itemize}
\subsection{Test Justification/Algnment}
\subsubsection{Default Alignment - Fully Justified}
By default text is fully justified - left and right margins are the same and text stretches to fill width of page
\medskip
sample text sample text sample text sample text sample text sample text sample text sample text sample text sample text sample text sample text sample text sample text sample text 
\subsubsection{Left Justified Text}
\begin{flushleft}
    Left justified text is aligned with left margin, words are not streched out, so right margin is ragged
    
    \medskip
sample text sample text sample text sample text sample text sample text sample text sample text sample text sample text sample text sample text sample text sample text sample text 
\end{flushleft}
\subsubsection{Center Justified Text}
\begin{center}
    Center justified text is aligned to center of the document, spacing between words is not streched out 
    so left and right margins are ragged 
    \medskip
sample text sample text sample text sample text sample text sample text sample text sample text sample text sample text sample text sample text sample text sample text sample text 
\end{center}    

\subsubsection{Right justified text}
\begin{flushright}
sample text sample text sample text sample text sample text sample text sample text sample text sample text sample text sample text sample text sample text sample text sample text 
\end{flushright}
\subsection{Line Breaks}
This is a line of text with \\ 
4 line  break line \\[4\baselineskip]
extra space and break line 
 
\subsection{Indentation}
Every paragraph is indented automatically. To create new paragraph we need to put two Enter keys

This is new, indented paragraph

\noindent This is paragraph without indentation

\section{Basic Math Manipulation}
\subsection{Math Modes  }
\subsubsection{Display Style Math}
Math equations are in center of page
\begin{itemize}
    \item Backslash bracket combination (best for single lines of math equations):
    \[f(x) = (x + 2)^2 - 9\]
    \item align*(star means equations won't be numbered   ) environment - alignes multiline equations
    the ampersand character \texttt{\&} determines where the equations align.
    
\begin{align*} 
2x - 5y &=  8 \\ 
3x + 9y &=  -12
\end{align*}
It also can be used to align multiple equations:
\begin{align*}
    a + b & = 0         & 3y - 2 & = 5 \\
    2a + 3 & = b        & 2x + 32 - 2 & = 11 \\
    b + 4a + 11 & = 7   & 21z * 2 - 3y & = 23
\end{align*}
Numbered equations: 
\begin{align}
    a + b & = 0         & 3y - 2 & = 5 \\
    2a + 3 & = b        & 2x + 32 - 2 & = 11 \\
    b + 4a + 11 & = 7   & 21z * 2 - 3y & = 23
\end{align}
Skippiing one number:
\begin{align}
    a + b & = 0         & 3y - 2 & = 5 \\
    2a + 3 & = b        & 2x + 32 - 2 & = 11 \nonumber \\
    b + 4a + 11 & = 7   & 21z * 2 - 3y & = 23
\end{align}
\end{itemize}
\subsubsection{Inline}
It is just math that stays in line. There are two ways of doing it (first is dated)
\begin{itemize}
    \item This is Pythagorean teorem: $a^2 + b^2 = c^2$ which you should know by now.
    \item This is Pythagorean teorem: \(a^2 + b^2 = c^2\) which you should know by now.
\end{itemize}

\subsubsection{More Differences in Math Modes}
Lets look at formula:
    \begin{equation*}
        \sum_{n=0}^\infty \frac{1}{n^2} = \frac{\pi^2}{6}
    \end{equation*}
* \uline{This is how it will look like in forced display mode inside text:}

text text text  text text text text text text text text ttexttexttexttexttexttextext text\(\displaystyle \sum_{n=0}^\infty \frac{1}{n^2} = \frac{\pi^2}{6}\) text text text text text text text text texttexttexttexttexttext
ext texttext text

\uline{* This is how it will look like in forced inline mode}

Text text  text text text text text text \[\textstyle \sum_{n=0}^\infty \frac{1}{n^2} = \frac{\pi^2}{6}\] text  text  text  text  text  text  text  text  text  text  text  text 

\subsection{Basic Notation}\label{subsec:basic-notation}

\subsubsection{Arithmetic}
\begin{itemize}
    \item Addition: 1 + 1
    \item Substraction: 1 - 1
    \item Multiplication:
    \begin{itemize}
        \item cdot: \(a^2 \cdot 2 \)
        \item times: \(a^2 \times 2\) 
    \end{itemize}
\end{itemize}

\subsubsection{Fractions}
Fraction in display style: \[\frac{numerator}{denominator}\]
\noindent
Fraction in inline style \(\frac{numerator}{denominator}\)
Forced fractions:
\begin{itemize}
    \item forced text-style (inline) \[\tfrac{numerator}{denominator}\]
    \item forced display-style \(\dfrac{numerator}{denominator}\)
\end{itemize}
 
\subsection{Superscript and subscript}
\begin{itemize}
\item Superscript: \(a ^ 2\)
\item Subscript: \(a_2\)
\item Grouped with brackets: \(e^{kx}\), without brackets: \(e^kx\)
\item Simultaneous Superscript and Subscript: \(a_1^2\), \(a^2_1\)
\item Combined Superscripts and Subscripts
\begin{itemize}
    \item Stacked Style: \(x_1^{y_1}\) 
    \item Offset Style: \({x_1}^{y^1}\)
\end{itemize}

\section{Parentheses}\label{sec:parentheses}
To autoresize parantheses we use \textbf{\textbackslash left} and \textbf{\textbackslash right}:
\[\left( \left( \frac{1}{a + b} \right)^2 \right)\]
    
\end{itemize}
some text
\end{document}

